\documentclass[12pt]{article}

\usepackage{amsmath, mathtools, amssymb}
\usepackage[margin=0.5in]{geometry}
\usepackage[document]{ragged2e}

\makeatletter
\renewcommand*\env@matrix[1][*\c@MaxMatrixCols c]{%
  \hskip -\arraycolsep
  \let\@ifnextchar\new@ifnextchar
  \array{#1}}
\makeatother

\begin{document}
UMEP Linear Algebra \hfill Felix McCuaig \\
Assignment 1

\section*{Question1}
For the system of equations:

\begin{align*}
\begin{aligned}
3tx + (t+12)y + (t-3)z &=& 3 \\
3x + (t-3)y + z &=& 3t \\
x+y+3z &=& 2 \\
\end{aligned} \hspace*{20px}
\begin{bmatrix}[ccc|c]
   3t & t+12 & t-3 & 3 \\
   3 & t-3 & 1 & 3t \\
   1 & 1 & 3 & 2 \\
\end{bmatrix}
\end{align*}

(a) Find the values of t for which the system is consistent.\\
\medskip
Display in augmented matrix form.
Use elementary operations for row echelon form.
$$
\begin{bmatrix}[ccc|c]
   3t & t+12 & t-3 & 3 \\
   3 & t-3 & 1 & 3t \\
   1 & 1 & 3 & 2 \\
\end{bmatrix} \overset{r_2-3r_3}{\longrightarrow} 
\begin{bmatrix}[ccc|c]
   3t & t+12 & t-3 & 3 \\
   0 & t-6 & -8 & 3t-6 \\
   1 & 1 & 3 & 2 \\
\end{bmatrix} \overset{r_3\cdot3t}{\longrightarrow} 
\begin{bmatrix}[ccc|c]
   3t & t+12 & t-3 & 3 \\
   0 & t-6 & -8 & 3t-6 \\
   3t & 3t & 9t & 6t \\
\end{bmatrix}
$$

$$\begin{bmatrix}[ccc|c]
   3t & t+12 & t-3 & 3 \\
   0 & t-6 & -8 & 3t-6 \\
   3t & 3t & 9t & 6t \\
\end{bmatrix} \overset{r_3-r_1}{\longrightarrow} 
\begin{bmatrix}[ccc|c]
   3t & t+12 & t-3 & 3 \\
   0 & t-6 & -8 & 3t-6 \\
   0 & 2t-12 & 8t+3 & 6t-3 \\
\end{bmatrix} \overset{r_3-2r_2}{\longrightarrow} 
\begin{bmatrix}[ccc|c]
   3t & t+12 & t-3 & 3 \\
   0 & t-6 & -8 & 3t-6 \\
   0 & 0 & 8t+19 & 9 \\
\end{bmatrix}
$$
Now it becomes apparent that if $t=\frac{-19}{8}$ the system is inconsistent. However, the other values of t for which the system is inconsistent are not so clear. Now if we let $A$ be the coefficient matrix, we can solve $Det(A)=0$ for $t$ we could find other values of $t$ for which the system is inconsistent. For this upper triangular matrix, we can simply multiply along the diagonal like so:
$$
A = 
\begin{bmatrix}
   3t & t+12 & t-3 \\
   0 & t-6 & -8  \\
   0 & 0 & 8t+19 \\
\end{bmatrix}
$$
$$
Det(A)=0 \hspace{20px}
\therefore
\begin{vmatrix}
   3t & t+12 & t-3\\
   0 & t-6 & -8\\
   0 & 0 & 8t+19\\
\end{vmatrix} = 0
\hspace*{20pt}
\therefore(8t+19)\cdot(3t)\cdot(t-6)=0 
\hspace*{20pt}
\therefore t=\frac{-19}{8}, 0, 6
$$
Hence, the system is consistent when $t \in \mathbb{R} \setminus \{\frac{-19}{8},0,6\}$ \\
\medskip
(b) Now, we can sub in our values for $t$ to determine if they will yield more than one solution.
$$
t = \frac{-19}{8}
\hspace*{20pt}
\begin{bmatrix}[ccc|c]
   3\cdot\frac{-19}{8} & \frac{-19}{8}+12 & \frac{-19}{8}-3 & 3 \\
   0 & \frac{-19}{8}-6 & -8 & 3\cdot\frac{-19}{8}-6 \\
   0 & 0 & 8\cdot\frac{-19}{8}+19 & 9 \\
\end{bmatrix}  
\sim
\begin{bmatrix}[ccc|c]
   3\cdot\frac{-19}{8} & \frac{-19}{8}+12 & \frac{-19}{8}-3 & 3 \\
   0 & \frac{-19}{8}-6 & -8 & 3\cdot\frac{-19}{8}-6 \\
   0 & 0 & 0 & 9 \\
\end{bmatrix}
$$
When $t=\frac{-19}{8}$ there is no solution.
$$
t = 6
\hspace*{20pt}
\begin{bmatrix}[ccc|c]
   3\cdot6 & 6+12 & 6-3 & 3 \\
   0 & 6-6 & -8 & 3\cdot6-6 \\
   0 & 0 & 8\cdot6+19 & 9 \\
\end{bmatrix}
\sim
\begin{bmatrix}[ccc|c]
  18 & 18 & 3 & 3 \\
   0 & 0 & -8 & 12 \\
   0 & 0 & 67 & 9 \\
\end{bmatrix}
\overset{r_2\cdot\frac{67}{8}}{\longrightarrow}
\begin{bmatrix}[ccc|c]
  18 & 18 & 3 & 3 \\
   0 & 0 & -67 & \frac{201}{2} \\
   0 & 0 & 67 & 9 \\
\end{bmatrix}
$$
\newpage
When $t=6$ there is no solution.
$$
t = 0
\hspace*{20pt}
\begin{bmatrix}[ccc|c]
   3\cdot0 & 0+12 & 0-3 & 3 \\
   0 & 0-6 & -8 & 3\cdot0-6 \\
   0 & 0 & 8\cdot0+19 & 9 \\
\end{bmatrix}
\sim
\begin{bmatrix}[ccc|c]
  0 & 12 & -3 & 3 \\
   0 & -6 & -8 & -6 \\
   0 & 0 & 19 & 9 \\
\end{bmatrix}
\overset{r_2\cdot2}{\longrightarrow}
\begin{bmatrix}[ccc|c]
   0 & 12 & -3 & 3 \\
   0 & -12 & -16 & -12 \\
   0 & 0 & 19 & 9 \\
\end{bmatrix}
$$

$$
\overset{r_1+r_2}{\longrightarrow}
\begin{bmatrix}[ccc|c]
  0 & 0 & -19 & -9 \\
   0 & -12 & -16 & -12 \\
   0 & 0 & 19 & 9 \\
\end{bmatrix} \overset{r_1+r_3}{\longrightarrow}
\begin{bmatrix}[ccc|c]
  0 & 0 & 0 & 0 \\
   0 & -12 & -16 & -12 \\
   0 & 0 & 19 & 9 \\
\end{bmatrix} \overset{r_2\div4}{\longrightarrow}
\begin{bmatrix}[ccc|c]
  0 & 0 & 0 & 0 \\
   0 & -3 & -4 & -3 \\
   0 & 0 & 19 & 9 \\
\end{bmatrix}
$$
$$
\begin{aligned}
-3y-4z&=&-3 \\
19z&=&9
\end{aligned}
\hspace*{20px}
\therefore z=\frac{9}{19}
\hspace*{20px}
\therefore -3y-4\cdot\frac{9}{19}= -3
\hspace*{20px}
\therefore -3y=\frac{-21}{19} 
\hspace*{20px}
\therefore y=\frac{7}{19}
$$
\section*{Question 2}
(a) For square matrices of the same size, if $AB=BA$ and $BC=CB$, then $AC=CA$. \\
True. \\
(b) For any square invertible matrix D. \\
(c) If the row-echelon form of a square matrix E has a row of zeros, then $Det(E)=0$.\\
$$
E = 
\begin{bmatrix}
   1 & 5 & 7 \\
   0 & 7 & 4  \\
   0 & 0 & 0 \\
\end{bmatrix}
$$

For square matrices in row echelon form such as $E$, the determinant can be calculated by multiplying down the diagonal, in the case of $E$: $1\cdot7\cdot0=0$. Therefore, for any square matrix, if a whole row is zeros, the determinant must be zero. \\
\begin{aligned}
(d) For the matrix
$$
F = 
\begin{bmatrix}
   s-1 & 1 & 2 \\
   s^2-1 & 3 & 2  \\
   s+1 & 1 & 2 \\
\end{bmatrix}
$$
there is more than one value of $s$ for which $Det(F)=0$.
\end{aligned}





















\end{document}