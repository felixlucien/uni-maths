\documentclass[12pt]{article}

\usepackage{amsmath, mathtools, amssymb}
\usepackage[margin=0.5in]{geometry}
\usepackage[document]{ragged2e}
\usepackage{tikz}
\usepackage{undertilde}

\makeatletter
\renewcommand*\env@matrix[1][*\c@MaxMatrixCols c]{%
  \hskip -\arraycolsep
  \let\@ifnextchar\new@ifnextchar
  \array{#1}}
\makeatother
\newcommand{\norm}[1]{\left\lVert#1\right\rVert}
\renewcommand{\Bigg}{\bBigg@{2.5}}

\begin{document}

UMEP Linear Algebra \hfill Felix McCuaig \\
Assignment 2

\section*{Question 1}

$$
A=
\begin{bmatrix}
   2 & 3 & 5 & 7\\
   3 & 5 & 7 & 9\\
   5 & 7 & 9 & 11\\
\end{bmatrix}
$$

(\textbf{a}) What is the rank of $A$.\\

$$
\begin{bmatrix}
   2 & 3 & 5 & 7\\
   3 & 5 & 7 & 9\\
   5 & 7 & 9 & 11\\
\end{bmatrix} \overset{2r_2-3r_1}{\longrightarrow} 
\begin{bmatrix}
   2 & 3 & 5 & 7\\
   0 & 1 & -1 & -3\\
   5 & 7 & 9 & 11\\
\end{bmatrix} \overset{2r_3-5r_1}{\longrightarrow} 
\begin{bmatrix}
    2 & 3 & 5 & 7\\
   0 & 1 & -1 & -3\\
   0 & -1 & -7 & -13\\
\end{bmatrix} \overset{r_3+r_2}{\longrightarrow} 
\begin{bmatrix}
    2 & 3 & 5 & 7\\
   0 & 1 & -1 & -3\\
   0 & 0 & -8 & -16\\
\end{bmatrix}
$$

It is clear that in row echelon form, there are $3$ pivots, the dimension of the row space is $3$ and therefore the rank is $3$.\\
\medskip
(\textbf{b}) Find a basis for the solution space of $A$.\\
$$
\begin{bmatrix}
   2 & 3 & 5 & 7\\
   3 & 5 & 7 & 9\\
   5 & 7 & 9 & 11\\
\end{bmatrix}
\begin{bmatrix}
   x_1 \\
   x_2 \\
   x_3 \\
   x_4 \\
\end{bmatrix}
=
\begin{bmatrix}
   0 \\
   0 \\
   0 \\
\end{bmatrix}
$$
As we have seen before, this matrix $A$ is reduced to:
$$
\begin{bmatrix}[cccc|c]
   2 & 3 & 5 & 7 & 0\\
   0 & 1 & -1 & -3 & 0\\
   0 & 0 & 1 & 2 & 0\\
\end{bmatrix}
\hspace{20pt}
\therefore
\begin{aligned}
&2x_1& + &3x_2& + &5x_3& &=& 0 \\
&& &x_2& - &x_3& - 3x_4 &=& 0 \\
&& && &x_3& + 2x_4 &=& 0
\end{aligned}
\hspace{20pt}
\therefore
x_3 = -2x_4
$$
By substituting $x_3=-2x_4$ back into the other equations we find we only have two remaining variables:

$$
\begin{aligned}
2x_1 + 3x_2 - 10x_4 &=& 0 \\
x_2 &=& 5x_4 \\
x_3 &=& -2x_4
\end{aligned}
\hspace{20pt}
\therefore
\begin{aligned}
2x_1 &=& -13x_4 \\
x_2 &=& 5x_4 \\
x_3 &=& -2x_4
\end{aligned}
$$
If we let $x_4 = a$ then:
$$
\begin{bmatrix}
   x_1 \\
   x_2 \\
   x_3 \\
   x_4 \\
\end{bmatrix}
=
\begin{bmatrix}
   \frac{-13}{2} \\
   5 \\
   -2 \\
   1 \\
\end{bmatrix}\cdot a
$$
Therefore a basis $B$ for the solution space of the matrix $A$ is:
$$
B = 
\left\{
\left(
\frac{-13}{2}, 5, -2, 1
\right)
\right\}
$$
\medskip
(\textbf{c}) Find a basis for the row space of $A$.\\
By row reducing $A$ as we have previously seen:
$$
\begin{bmatrix}
   2 & 3 & 5 & 7 \\
   3 & 5 & 7 & 9 \\
   5 & 7 & 9 & 11 \\
\end{bmatrix}
\sim
\begin{bmatrix}
   2 & 3 & 5 & 7 \\
   0 & 1 & -1 & -3 \\
   0 & 0 & -8 & -16 \\
\end{bmatrix}
\hspace{20pt}
\therefore
B = 
\left\{
\begin{bmatrix}
   2 \\
   3 \\
   5 \\
   7 \\
\end{bmatrix}
,
\begin{bmatrix}
   0 \\
   1 \\
   -1 \\
   -3 \\
\end{bmatrix}
,
\begin{bmatrix}
   0 \\
   0 \\
   -8 \\
   -16 \\
\end{bmatrix}
\right\}
$$
\newpage
(\textbf{d}) State the theorem relating the dimensions of the solution space to the rank of the matrix. Verify that this theorem holds for $A$.\\
\medskip
The rank-nullity theorem states that: $Rank(A)+Nullity(A)=dim\:V$. In the case of $A$:
$$
Rank(A)=3,\;Nullity(A)=1,\;3+1=4=dim\:V
$$

\section*{Question 2}

Let $V=(-1,1)=\{x \in \mathbb{R}\hspace{5px}| -1 < x < 1 \}$, and define the operations \\
$$
x \ast y = \frac{x+y}{1+xy}
\hspace{100pt}
\text{(vector addition)}
$$ 

$$
a \odot x = \frac{(1+x)^{\alpha} - (1-x)^{\alpha}}{(1+x)^{\alpha} + (1-x)^{\alpha}}
\hspace{60pt}
\text{(scalar multiplication)}
$$
Where $x,y \in V$ and $a \in \mathbb{R}$. Determine whether $V$, with the operations listed is a vector space over $\mathbb{R}$.\\
\medskip
\textbf{Axiom 1} ($x+y$ is in $V$)
$$
x\ast y=\frac{x+y}{(1-x)(1-y) + x + y}
$$
\begin{center}
%For $x,y \in (-1,1)$\hspace{120pt} (fulfills axiom 1)\\
As $x+y \to 2$, $x\ast y \to 1$.\\
As $x+y \to -2$, $x\ast y \to -1$.\\
\end{center}
\medskip
\textbf{Axiom 2} ($x+y = y+x$)
$$
x\ast y=\frac{x+y}{1+x\cdot y}=\frac{y+x}{1+y\cdot x}=y\ast x
\hspace{60pt}
\text{(fulfills axiom 2)}
$$
\medskip
\textbf{Axiom 3} $(u+v)+w = u+(v+w)$
$$
(u \ast v) \ast w = \left(\frac{u + v}{1 + u \cdot v}\right)\ast w 
= \frac{\frac{u+v}{1+ uv} + w}{1 + w\frac{u+v}{1+ uv}}
= \frac{\frac{u+v}{1+uv} + w\frac{1+uv}{1+uv}}{\frac{1+uv}{1+uv} + w \frac{u+v}{1+uv}}
= \frac{u+v+w+uvw}{1+uv+wu+wv}
= \frac{v+w+u(1+wv)}{1+wv+u(w+v)}
$$

$$
\therefore
\frac{\frac{v+w}{1+wv} + v}{1+u\frac{w+v}{1+vw}}=u\ast \frac{w+v}{1+wv}
=u\ast (w \ast v)
\hspace{40px}
\text{(fulfills axiom 3)}
$$
\textbf{Axiom 4} There is a zero vector in $V$ where $u+0=u$
$$
0 \ast u = \frac{u+0}{1 + u\cdot 0}= \frac{u}{1}= u
\hspace{110px}
\text{(fulfills axiom 4)}
$$
\textbf{Axiom 5} For each vector $u$ in $V$ there is a vector $-u$ satisfying $u+(-u)=0$.
$$
u\ast-u=\frac{u-u}{1+u^2}=\frac{0}{1=u^2}=0
\hspace{90pt}
\text{(fulfills axiom 5)}
$$
\textbf{Axiom 6} $\alpha\odot x$ is in $V$
$$\text{As } x \to 1,
\hspace{20pt} (1+x)^\alpha\to2^\alpha
\hspace{20pt} (1-x)^\alpha\to0
\hspace{20pt} \therefore
\frac{2^\alpha-0}{2^\alpha + 0}\to1
$$

$$\text{As } x \to -1,
\hspace{20pt} (1+x)^\alpha\to0
\hspace{20pt} (1-x)^\alpha\to2^\alpha
\hspace{20pt} \therefore
\frac{0-2^\alpha}{0 + 2^\alpha}\to-1
$$

$$
\text{When } x=0,\hspace{20pt} \frac{(1)^\alpha-(1)^\alpha}{(1)^\alpha+(1)^\alpha}=0. \hspace{70pt} \text{(fulfills axiom 6)}
$$
\newpage
\textbf{Axiom 7} $\alpha \odot(u+v)$ = $\alpha \odot u + \alpha \odot v$\\
Let $u=\frac{1}{4}$ and $v=\frac{1}{2}$:\\
$$
u \ast v =
\frac{\frac{1}{2}+\frac{1}{4}}{1+\frac{1}{2}\cdot \frac{1}{4}} 
=\frac{\frac{3}{4}}{\frac{9}{8}}
=\frac{2}{3}
$$
Now, let $\alpha=4$:
$$
4
\odot
\left(\frac{1}{2}\ast \frac{1}{4}\right)
=
\frac{
\left(1+\frac{2}{3}\right)^4-\left(1-\frac{2}{3}\right)^4
}{
\left(1+\frac{2}{3}\right)^4+\left(1-\frac{2}{3}\right)^4
}
=
\frac{312}{313}
$$
Now, if we take $\alpha\odot \frac{1}{2} + \alpha\odot \frac{1}{4}$:

$$
4\odot \frac{1}{2}
+
4\odot \frac{1}{4}
=
\frac{
\left(
1+\frac{1}{2}
\right)^4
-
\left(
1-\frac{1}{2}
\right)^4
}{
\left(
1+\frac{1}{2}
\right)^4
+
\left(
1-\frac{1}{2}
\right)^4
}
+
\frac{
\left(
1+\frac{1}{4}
\right)^4
-
\left(
1-\frac{1}{4}
\right)^4
}{
\left(
1+\frac{1}{4}
\right)^4
+
\left(
1-\frac{1}{4}
\right)^4
}
=\frac{25272}{14473}
$$
Clearly, they are not equal and therefore $V$ doesn't fulfil the requirements to be a vector space.
\section*{Question 3}
(\textbf{a}) Find a subset of the set
$$
\mathcal{S}=\{(0, -1, -3, 3), (-1, -1, -3, 2), (3, 1, 3, 0), (0, -1, -2, 1)\}
$$
of vectors in $\mathbb{R}^4$, that form a basis for $Span(\mathcal{S})$.\\
\medskip
The basis vectors can be described as a matrix:\\
$$
\mathcal{S}=
\begin{bmatrix}
   0 & -1 & -3 & 3 \\
   -1 & -1 & -3 & 2 \\
   3 & 1 & 3 & 0 \\
   0 & -1 & -2 & 1
\end{bmatrix}
$$
When row-reduced this becomes:
$$
\sim
\begin{bmatrix}
   1 & \frac{1}{3} & 1 & 0 \\
   0 & 1 & 3 & -3 \\
   0 & 0 & 1 & -2 \\
   0 & 0 & 0 & 0
\end{bmatrix}
\sim
\begin{bmatrix}
   1 & 0 & 0 & 1 \\
   0 & 1 & 0 & 3 \\
   0 & 0 & 1 & -2 \\
   0 & 0 & 0 & 0
\end{bmatrix}
$$
Therefore a subset of $\mathcal{S}$, designated as basis $\mathcal{B}$ which forms a subspace of $\mathbb{R}^3$ in $\mathbb{R}^4$ and is a basis for $Span(\mathcal{S})$ is:
$$
\mathcal{B}=\left\{
\begin{bmatrix}
   1 \\
   0 \\
   0 \\
   1
\end{bmatrix},
\begin{bmatrix}
   0 \\
   1 \\
   0 \\
   3
\end{bmatrix},
\begin{bmatrix}
   0 \\
   0 \\
   1 \\
   -2
\end{bmatrix}
\right\}
$$
(\textbf{b}) Show that the set
$$
\mathcal{B}=\{(-1,-3,-8,6),(-1,0,-1,1),(-1,-2,-6,5)\}
$$
is another basis for $Span(\mathcal{S})$.\\
\medskip
$\mathcal{B}$ can be described in matrix form and let this new basis be $\mathcal{M}$:
$$
\mathcal{M}=
\begin{bmatrix}
   -1 & -3 & -8 & 6 \\	
   -1 & -0 & -1 & 1 \\
   -1 & -2 & -6 & 5 \\
\end{bmatrix}
\sim
\begin{bmatrix}
   1 & 0 & 0 & 1 \\	
   0 & 1 & 0 & 3 \\
   0 & 0 & 1 & -2 \\
\end{bmatrix}
$$
$$
\mathcal{M}=
\left\{
\begin{bmatrix}
   1 \\
   0 \\
   0 \\
   1
\end{bmatrix},
\begin{bmatrix}
   0 \\
   1 \\
   0 \\
   3
\end{bmatrix},
\begin{bmatrix}
   0 \\
   0 \\
   1 \\
   -2
\end{bmatrix}
\right\}
$$
It is clear this is similar to the result we found in (\textbf{a}), this set is indeed another basis for $Span(\mathcal{S})$.
(\textbf{c}) Let $\mathbf{v}=(3,7,20,-16)\in \mathbb{R}^4$. Write the coordinate vector of $\mathbf{v}$ with respect to\\
\hspace{40px}(i) the basis found in part (\textbf{a});
$$
\begin{bmatrix}
   1 & 0 & 0 \\
   0 & 1 & 0 \\
   0 & 0 & 1 \\
   1 & 3 & -2
\end{bmatrix}
\begin{bmatrix}
   a \\
   b \\
   c
\end{bmatrix}
=
\begin{bmatrix}
 3 \\
 7 \\
 20 \\
 -16
\end{bmatrix}
\sim
\begin{bmatrix}[ccc|c]
    1 & 0 & 0 & 3 \\
   0 & 1 & 0 & 7\\
   0 & 0 & 1 & 20\\
   1 & 3 & -2 & -16
\end{bmatrix}
\sim
\begin{bmatrix}[ccc|c]
    1 & 0 & 0 & 3 \\
   0 & 1 & 0 & 7\\
   0 & 0 & 1 & 20\\
   0 & 0 & 0 & 0
\end{bmatrix}
\therefore a=3, b=7,c=20
$$

$$
\therefore \mathbf{v}=
3
\begin{bmatrix}
   1 \\
   0 \\
   0 \\
   1
\end{bmatrix}
+7
\begin{bmatrix}
   0 \\
   1 \\
   0 \\
   3
\end{bmatrix}
+20
\begin{bmatrix}
   0 \\
   0 \\
   1 \\
   -2
\end{bmatrix}
$$
\hspace{40pt}
(ii) the basis $\mathcal{B}$.\\
$$
\mathbf{v}=
a
\begin{bmatrix}
   -1 & -1 & -1 \\
   -3 & 0 & -2 \\
   -8 & -1 & -6 \\
   0 & 1 & 5
\end{bmatrix}
\begin{bmatrix}
	a \\
	b \\
	c
\end{bmatrix}
=
\begin{bmatrix}
   3 \\
   7 \\
   20 \\
   -16
\end{bmatrix}
\sim
\begin{bmatrix}[ccc|c]
   -1 & -1 & -1 & 3 \\
   -3 & 0 & -2 & 7 \\
   -8 & -1 & -6 & 20 \\
   6 & 1 & 5 & -16 \\
\end{bmatrix} \overset{r_4+2r_2}{\longrightarrow} 
\begin{bmatrix}[ccc|c]
   -1 & -1 & -1 & 3 \\
   -3 & 0 & -2 & 7 \\
   -8 & -1 & -6 & 20 \\
   0 & 1 & 1 & -2 \\
\end{bmatrix} \overset{r_2-3r_1}{\longrightarrow}
$$

$$
\begin{bmatrix}[ccc|c]
   -1 & -1 & -1 & 3 \\
   0 & 3 & 1 & -5 \\
   -8 & -1 & -6 & 20 \\
   0 & 1 & 1 & -2 \\
\end{bmatrix} \overset{r_3-8r_1}{\longrightarrow}
\begin{bmatrix}[ccc|c]
   -1 & -1 & -1 & 3 \\
   0 & 3 & 1 & -5 \\
   0 & 7 & 2 & -4 \\
   0 & 1 & 1 & -2 \\
\end{bmatrix} \overset{r_3-2r_4}{\longrightarrow}
\begin{bmatrix}[ccc|c]
   -1 & -1 & -1 & 3 \\
   0 & 3 & 1 & -5 \\
   0 & 5 & 0 & 0 \\
   0 & 1 & 1 & -2 \\
\end{bmatrix} \overset{r_2-r_4}{\longrightarrow}
\begin{bmatrix}[ccc|c]
   -1 & -1 & -1 & 3 \\
   0 & 2 & 0 & -3 \\
   0 & 5 & 0 & 0 \\
   0 & 1 & 1 & -2 \\
\end{bmatrix}
$$

$$
 \overset{2r_4-r_2}{\longrightarrow}
\begin{bmatrix}[ccc|c]
   -1 & -1 & -1 & 3 \\
   0 & 1 & 0 & 0 \\
   0 & 0 & 1 & -2 \\
   0 & 2 & 0 & -3 \\
\end{bmatrix} \overset{r_1+r_2}{\longrightarrow}
\begin{bmatrix}[ccc|c]
   -1 & 0 & -1 & 3 \\
   0 & 1 & 0 & 0 \\
   0 & 0 & 1 & -2 \\
   0 & 2 & 0 & -3 \\
\end{bmatrix} \overset{r_1+r_3}{\longrightarrow}
\begin{bmatrix}[ccc|c]
   -1 & 0 & 0 & 1 \\
   0 & 1 & 0 & 0 \\
   0 & 0 & 1 & -2 \\
   0 & 2 & 0 & -3 \\
\end{bmatrix}
\therefore
a=-1, b=0, c=-2
$$

To test these values for $a,b,c$ we can back-substitute them into the original equation.
$$
\mathbf{v}=
-1
\begin{bmatrix}
   -1 \\
   -3 \\
   -8 \\
   6
\end{bmatrix}
+0
\begin{bmatrix}
   -1 \\
   0 \\
   -1 \\
   1
\end{bmatrix}
-2
\begin{bmatrix}
   -1 \\
   -2 \\
   -6 \\
   5
\end{bmatrix}
=
\begin{bmatrix}
   1 + 2 \\
   3 + 4 \\
   8 + 12 \\
   -6 -10
\end{bmatrix}
=
\begin{bmatrix}
   3 \\
   7 \\
   20 \\
   -16
\end{bmatrix}
$$
(\textbf{d}) Consider the set of polynomials
$$
\mathcal{P}=\{-1-3x-8x^2+6x^3,\quad -1-x^2+x^3,\quad -1-2x-6x^2+5x^3\}
$$
Can the polynomial $p(x)=-3-5x-15x^2+12x^3$ be written as a linear combination of the polynomials in $\mathcal{P}$ in more than one way? 
$$
\begin{bmatrix}
   -1 & -1 & -1\\
   -3 & 0 & -2\\
   -8 & -1 & -6\\
   6 & 1 & 5
\end{bmatrix}
\begin{bmatrix}
  a \\
  b \\
  c
\end{bmatrix}
=
\begin{bmatrix}
   -3 \\
   -5 \\
   -15 \\
   12 
\end{bmatrix}
\sim
\begin{bmatrix}[ccc|c]
   -1 & -1 & -1 & -3 \\
   -3 & 0 & -2 & -5 \\
   -8 & -1 & -6 & -15 \\
   6 & 1 & 5 & 12 \\
\end{bmatrix} \overset{r_4+2r_2}{\longrightarrow}
\begin{bmatrix}[ccc|c]
   -1 & -1 & -1 & -3 \\
   -3 & 0 & -2 & -5 \\
   -8 & -1 & -6 & -15 \\
   0 & 1 & 1 & 2 \\
\end{bmatrix}
$$

$$
\overset{r_2-3r_1}{\longrightarrow}
\begin{bmatrix}[ccc|c]
   -1 & -1 & -1 & -3 \\
   -3 & 0 & -2 & -5 \\
   -8 & -1 & -6 & -15 \\
  0 & 1 & 1 & 2 \\
\end{bmatrix} \overset{r_3-8r_1}{\longrightarrow}
\begin{bmatrix}[ccc|c]
   -1 & -1 & -1 & -3 \\
   -3 & 0 & -2 & -5 \\
   0 & 7 & 2 & 9 \\
   0 & 1 & -3 & 12 \\
\end{bmatrix} \overset{r_2-3r_1}{\longrightarrow}
\begin{bmatrix}[ccc|c]
   -1 & -1 & -1 & -3 \\
   0 & 3 & 1 & 4\\
   0 & 7 & 2 & 9 \\
   0 & 1 & 1 & 2 \\
\end{bmatrix}
$$

$$
\overset{7r_2-3r_3}{\longrightarrow}
\begin{bmatrix}[ccc|c]
   -1 & -1 & -1 & -3 \\
   0 & 0 & 1 & 1 \\
   0 & 7 & 2 & 9 \\
   0 & 1 & 1 & 2 \\
\end{bmatrix} \overset{r_3-2r_2}{\longrightarrow}
\begin{bmatrix}[ccc|c]
   -1 & -1 & -1 & -3 \\
   0 & 0 & 1 & 1 \\
   0 & 7 & 0 & 7 \\
  0 & 1 & 1 & 2 \\
\end{bmatrix} \overset{r_4-(r_2+r_3)}{\longrightarrow}
\begin{bmatrix}[ccc|c]
   1 & 1 & 1 & 3 \\
   0 & 1 & 0 & 1\\
   0 & 0 & 1 & 1 \\
   0 & 0 & 0 & 0 \\
\end{bmatrix}
$$
$$
\overset{r_1-(r_2+r_3)}{\longrightarrow}
\begin{bmatrix}[ccc|c]
   1 & 0 & 0 & 1 \\
   0 & 1 & 0 & 1\\
   0 & 0 & 1 & 1 \\
   0 & 0 & 0 & 0 \\
\end{bmatrix}
\therefore a=1, b=1, c=1
$$
To check that this satisfies $p(x)$, we can substitute the values into the original equation:
$$
\begin{bmatrix}
   -1 \\
   -3 \\
   -8 \\
   6
\end{bmatrix}
+
\begin{bmatrix}
   -1 \\
   0 \\
   -1 \\
   1
\end{bmatrix}
+
\begin{bmatrix}
   -1 \\
   -2 \\
   -6 \\
   6
\end{bmatrix}
=
\begin{bmatrix}
   -3 \\
   -5 \\
   -15 \\
   12 \\
\end{bmatrix}
= p(x)
$$
Since the equation above has a single solution, there is only one way $p(x)$ can be written in terms of $\mathcal{P}$.
\section*{Question 4}
Let $V$ be a finite dimensional vector space over $\mathbb{R}$ with an inner product $\langle-,-\rangle : V \times V \rightarrow V$.\\
Let $W$ be a subspace of $V$. Define the set\\
$$
W^\bot=\{v \in V \: | \: \langle v,w \rangle = 0 \text{ for all } w \in W \}
$$\\
\medskip
(\textbf{a}) Show that $W^\bot$ is a subspace of $V$.\\
\medskip
Let $\{v_1,v_2,w_1\}\in   W^\bot$
$$
\textbf{(1)}
\hspace{20pt}
\langle
v_1 + v_2, w_1
\rangle
=
\langle
v_1, w_1
\rangle
+
\langle
v_2, w_1
\rangle
=
0+0=0
\hspace{40pt}
\therefore
\text{(Closed under addition)}
$$

$$
\textbf{(2)}
\hspace{20pt}
\alpha \cdot \langle v_1, w_1 \rangle = \alpha \cdot 0 = 0 \hspace{130pt} \therefore \text{(Closed under multiplication)}
$$

$$
\textbf{(3)}
\hspace{20pt}
\langle v_1, w_1 \rangle = 0 \hspace{270pt} \therefore \text{(Non empty)}
$$ \\
\medskip
(\textbf{b}) Prove that $dim\;W + dim\;W^\bot = dim\;V$ \\
\medskip
Recalling the rank nullity theorem from earlier:
$$
	rank\;A+Nullity\;A=dim\;V
$$
Where $V$ is some vector space over $\mathbb{R}^n$.\\
Suppose we create a matrix $B$ which spans $W$.
$$
	B=
	\begin{bmatrix}
		\mathbf{v_1} & \mathbf{v_2} & \cdots & \mathbf{v_n} 	
	\end{bmatrix}
$$

$$
	\therefore rank\;B^T+nullity\;B^T=dim\;V
$$
But $rank\;B^T=rank\;B$, so:\\
$$
	\therefore rank\;B +nullity\;B^T=dim\;V
$$

Rank of $A$ is just the dimension of the column space of $A$ and nullity the dimension of the null space so:\\
$$
	\therefore dim\;C(A)+dim\;N(A^T)=dim\;V
$$
We can recall that $C(A)$ spans $W$ and $N(A^T)=C(A)^\bot=W^\bot$.
$$
	\therefore dim\; W + dim\;W^\bot =dim \; V
$$

\end{document}