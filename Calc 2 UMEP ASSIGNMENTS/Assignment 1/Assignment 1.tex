\documentclass[12pt]{article}
\usepackage{amsmath, mathtools, amssymb, stmaryrd}
\usepackage[margin=0.5in]{geometry}
\usepackage[document]{ragged2e}
\usepackage{graphicx}
\usepackage{multicol}

\DeclareMathOperator{\arccosh}{arccosh}
\DeclareMathOperator{\arcsinh}{arcsinh} 

\begin{document}
UMEP Calculus \hfill Felix McCuaig \\
Assignment 1

\section*{Question 1}
\textbf{Consider the function:}
$$
f(x)= 
\begin{cases}
 \frac{\log(1-2x)}{x} & x < 0 \\
 a & x = 0 \\
 x^2\cos\left( \frac{1}{x}\right) & x > 0 \\
\end{cases}
$$
\textbf{For which values of $a, b \in \mathbb{R}$ is $f$ continuous at $x=0$?}\\
\medskip
At the point $x=0$, the following should be true for $f$ to be continuous:\\
$$
\lim_{x\rightarrow 0^-}\frac{\log(1-2x)}{x}=\lim_{x\rightarrow 0}a=\lim_{x\rightarrow 0^+}x^2\cos\left( \frac{1}{x} \right)
$$
We can't just take the limits of $f$ when $x>0$ and $x<0$ directly, we need to use some tricks first.\\
\smallskip
For $f(x)$ when $x<0$, we can use l'hopital's rule, take the derivative of both the top and the bottom and happy days:
$$
\lim_{x \rightarrow 0^-}\frac{\log(1-2x)}{x}=\lim_{x \rightarrow 0^-}\frac{\frac{-2}{1-2x}}{1}=\lim_{x \rightarrow 0^-}\frac{-2}{1-2x}=-2
$$ 
When $x=0$, evaluating the limit yields:
$$
\lim_{x\rightarrow 0}a=a
$$
Finally to evaluate when $x>0$, we can use the sandwich theorem:
$$
\lim_{x \rightarrow 0^+}x^2\cos \left( \frac{1}{x} \right)+b
$$ 
$$
x^2+b \geq x^2\cos \left( \frac{1}{x} \right)+b \geq -x^2+b
$$
Then:
$$
\lim_{x \rightarrow 0^+}x^2+b=\lim_{x \rightarrow 0^+}x^2\cos \left( \frac{1}{x} \right)+b=\lim_{x \rightarrow 0^+}-x^2+b=b
$$
Therefore, combining our results, we find that:
$$
-2=a=b
$$
So
$$
a = -2 \hspace{20px} b = -2
$$
\section*{Question 2}
\textbf{Evaluate the following integrals:}\\
\medskip
\textbf{(a)}\\
$$
\int (\log x)^2 dx
$$
Using integration by parts we find:\\
$$
\int u \ dv = uv - \int v \ du
$$

$$
u=\log^2 x \hspace{20px} du=\frac{2\log x}{x} \text{ and } dv=1 \ dx \hspace{20px} v=x
$$

$$
\therefore \int \log^2 x \ dx = x \log^2 x - \int 2\log x \ dx
$$
Next, we must take the integral:
$$
\int \log x \ dx
$$
So:
$$
u = \log x \hspace{20px} du = \frac{1}{x} \text{ and } dv = \ dx \hspace{20px} v = x 
$$
$$
\therefore \int \log x = x \log x - x
$$
Therefore, if we put all the parts together we find that:
\begin{align*}
	\int \log^2 x \ dx =& x \log ^2 x - 2(x \log x - x) + C\\
	=& x \log ^2 x - 2x \log x + 2x + C\\
	=& x(\log ^2 x -2 \log x + 2) + C
\end{align*}
Where $C \in \mathbb{R}$.\\
\bigskip
\textbf{(b)}\\
$$
\int_{0}^{1}\frac{\cosh x}{\sinh^2 x + 2 \sinh x + 2}dx
$$
Since:
$$
\frac{dy}{dx}\sinh x=\cosh x
$$
We can apply the substitution let $u=\sinh x$, $du = \cosh x \ dx$.\\
Don't forget the terminals: $\sinh 1$ and $\sinh 0=0$
\begin{align*}
	=&\int_{0}^{\sinh 1}\frac{1}{u^2+2u+2}du\\
	=&\int_{0}^{\sinh 1}\frac{1}{(u+1)^2+1}du\\
	=&\left[ \arctan(u+1) \right]^{\sinh 1}_{0}\\
	=& \arctan(\sinh 1 +1) - \frac{\pi}{4}
\end{align*}
\section*{Question 3}
\textbf{(a) Show that}
$$
\int \sec^n x \ dx = \frac{1}{n - 1} \tan x \sec^{n-2} x + \frac{n-2}{n-1}\int \sec^{n-2} x \ dx
$$
We can use integration by parts to determine the reduction of $\int \sec^n x \ dx$:
$$
\int \sec^n x \ dx = \int \sec^2 x \sec^{n-2}x \ dx
$$
If we let:
$$
du = \sec^2 x \ dx \hspace{20px} u = \tan x \text{ and } v = \sec^{n-2}x \hspace{20px} dv= (n-2)\sin x \cos x \sec^n x \ dx
$$
$$
\therefore \int \sec^n x \ dx = \tan x \sec^{n-2} x - \int \tan x (n-2)\sin x \cos x \sec^n x \ dx
$$
$$
\therefore \int \sec^n x \ dx = \tan x \sec^{n-2} x - (n-2) \int \frac{\sin^2 x}{\cos^n x} \ dx
$$
Recall that $\sin^2 x + \cos^2 x = 1$ then:
$$
\therefore \int \sec^n x \ dx = \tan x \sec^{n-2} x - (n-2) \int \frac{1 - \cos^2 x}{\cos^n x} \ dx
$$

$$
\therefore \int \sec^n x \ dx = \tan x \sec^{n-2} x - (n-2) \left[ \int \frac{1}{\cos^n x} \ dx - \int \frac{\cos^2 x}{\cos^n x} \ dx \right]
$$

$$
\therefore \int \sec^n x \ dx = \tan x \sec^{n-2} x - (n-2) \left[ \int \sec^n x \ dx - \int \sec^{n-2} x \ dx \right]
$$

$$
\therefore \int \sec^n x \ dx = \tan x \sec^{n-2} x - (n-2)\int \sec^n x \ dx + (n-2)\int \sec^{n-2} x \ dx
$$
Rearrange to get all our $\sec^n x$ to one side:
$$
\therefore \int \sec^n x \ dx + (n-2)\int \sec^n x \ dx  = \tan x \sec^{n-2} x + (n-2)\int \sec^{n-2} x \ dx
$$
And factorize to yield:
$$
\therefore \int \sec^n x \ dx (n-2+1) = \tan x \sec^{n-2} x + (n-2)\int \sec^{n-2} x \ dx
$$
Which yields our final formula:
$$
\therefore \int \sec^n x \ dx = \frac{1}{n-1}\tan x \sec^{n-2} x + \frac{n-2}{n-1}\int \sec^{n-2} x \ dx
$$
\textbf{(b) Evaluate}\\
$$
\int \sec^5 x \ dx
$$
Use the formula from \textbf{(a)}:
$$
\int \sec^5 x \ dx= \frac{1}{4}\tan x \sec^3 x + \frac{3}{4}\int \sec^3 x \ dx
$$
Apply again to $\sec^3 x$:
$$
\int \sec^3 x \ dx= \frac{1}{2}\tan x \sec x + \frac{3}{4}\int \sec x \ dx
$$
We know:
$$
\int \sec x \ dx = \log |\sec x + \tan x | + C
$$
Therefore:
$$
\int \sec^5 x \ dx = \frac{1}{4}\tan x \sec^3 x + \frac{3}{8}\tan x \sec x + \frac{3}{4}\log |\sec x + \tan x | + C
$$
Where $C \in \mathbb{R}$.
\section*{Question 4}
\textbf{Use an appropriate substitution to evaluate the following integral}
$$
\int \sqrt{x^2+4x+3} \ dx
$$
First let's express as a perfect square:
$$
\int \sqrt{(x+2)^2-1} \ dx
$$
Now, let $x+2=\cosh u$
$$
\frac{dx}{du}=\sinh u
$$
$$
\int \sqrt{\cosh^2 u -1} \sinh u \ du
$$
Remembering that $\cosh^2 u - \sinh^2 u=1$:
$$
\int \sqrt{\sinh^2 u} \sinh u \ du
$$

$$
\int \sinh^2 u \ du
$$
Rearranging the formula $\cosh 2u = 1 + 2\sinh^2 u$:
$$
\int \frac{\cosh 2u -1}{2} \ du = \frac{1}{2}\int \cosh 2u -1 \ du
$$
$$
\frac{1}{2}\left( \frac{1}{2}\sinh 2u - u \right)+C
$$
Recall that $u=\arccosh(x+2)$\\
$$
\frac{1}{4}\sinh (2\arccosh(x+2))-\frac{\arccosh(x+2)}{2}+C
$$
Where $C \in \mathbb{R}$.
\end{document}